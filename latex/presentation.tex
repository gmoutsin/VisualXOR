\documentclass{beamer}
\usepackage{amsmath}
\usepackage{mathtools}
\usepackage{amssymb}
\usepackage{amsfonts}
\usepackage{tikz}
\usepackage{pgfplots}
\usepackage{xcolor}
\pgfplotsset{compat=1.11}


\definecolor{ceruleanblue}{rgb}{0.16, 0.32, 0.75}
\definecolor{darkcyan}{rgb}{0.0, 0.55, 0.55}
\definecolor{darkmagenta}{rgb}{0.55, 0.0, 0.55}
\definecolor{carminered}{rgb}{1.0, 0.0, 0.22}


\newcommand{\coloredOne}{{\color{carminered} 1}}
\newcommand{\coloredZero}{{\color{ceruleanblue} 0}}

%
% Choose how your presentation looks.
%
% For more themes, color themes and font themes, see:
% http://deic.uab.es/~iblanes/beamer_gallery/index_by_theme.html
%

\pdfsuppresswarningpagegroup=1 

\mode<presentation>
{
  \usetheme{default}      % or try Darmstadt, Madrid, Warsaw, ...
  \usecolortheme{default} % or try albatross, beaver, crane, ...
  \usefonttheme{default}  % or try serif, structurebold, ...
  \setbeamertemplate{navigation symbols}{}
  \setbeamertemplate{caption}[numbered]
} 

\usepackage[english]{babel}
\usepackage[utf8x]{inputenc}

\usepackage[
    type={CC},
    modifier={by-nc-sa},
    version={3.0},
]{doclicense}


\setlength\parindent{0pt}
\setlength\parskip{1em}

\newcommand{\pxWhite}{
 {\mathchoice
  {\includegraphics[height=1.6ex]{pics/PixelWhite}}
  {\includegraphics[height=1.6ex]{pics/PixelWhite}}
  {\includegraphics[height=1.2ex]{pics/PixelWhite}}
  {\includegraphics[height=0.9ex]{pics/PixelWhite}}
 }
}

\newcommand{\pxDGrey}{
 {\mathchoice
  {\includegraphics[height=1.6ex]{pics/PixelDarkGrey}}
  {\includegraphics[height=1.6ex]{pics/PixelDarkGrey}}
  {\includegraphics[height=1.2ex]{pics/PixelDarkGrey}}
  {\includegraphics[height=0.9ex]{pics/PixelDarkGrey}}
 }
}

\newcommand{\LL}{\mathord{\includegraphics[height=1.6ex]{pics/L}}}
\newcommand{\RR}{\mathord{\includegraphics[height=1.6ex]{pics/R}}}
\newcommand{\BB}{\mathord{\includegraphics[height=1.6ex]{pics/B}}}

% \newcommand{\pxWhite}{\mathord{\includegraphics[height=1.6ex]{pics/PixelWhite}}}
\newcommand{\pxBlack}{\mathord{\includegraphics[height=1.6ex]{pics/PixelBlack}}}
\newcommand{\pxLGrey}{\mathord{\includegraphics[height=1.6ex]{pics/PixelLightGrey}}}
% \newcommand{\pxDGrey}{\mathord{\includegraphics[height=1.6ex]{pics/PixelDarkGrey}}}

\newcommand{\smKeyGrey}{\mathord{\includegraphics[height=10ex]{pics/smallKeyGrey}}}
\newcommand{\smMesGrey}{\mathord{\includegraphics[height=10ex]{pics/smallPlaintextGrey}}}
\newcommand{\smCyphGrey}{\mathord{\includegraphics[height=10ex]{pics/smallCyphertextGrey}}}

\newcommand{\smKeyBlack}{\mathord{\includegraphics[height=10ex]{pics/smallKeyBlack}}}
\newcommand{\smMesBlack}{\mathord{\includegraphics[height=10ex]{pics/smallPlaintextBlack}}}
\newcommand{\smCyphBlack}{\mathord{\includegraphics[height=10ex]{pics/smallCyphertextBlack}}}

\newcommand{\smXORmesGrey}{\mathord{\includegraphics[height=10ex]{pics/smallXOR_Key_Cypher_Grey}}}
\newcommand{\smXORcyphGrey}{\mathord{\includegraphics[height=10ex]{pics/smallXOR_Key_Text_Grey}}}


\newcommand{\smUpKeyBlack}{\mathord{\includegraphics[height=10ex]{pics/SmallUpKeyBlack}}}
\newcommand{\smUpMesBlack}{\mathord{\includegraphics[height=10ex]{pics/SmallUpPlaintextBlack}}}
\newcommand{\smUpCyphBlack}{\mathord{\includegraphics[height=10ex]{pics/SmallUpCyphertextBlack}}}


\newcommand{\smUpKeyColored}{\mathord{\includegraphics[height=10ex]{pics/SmallUpKeyColored}}}
\newcommand{\smUpMesColored}{\mathord{\includegraphics[height=10ex]{pics/SmallUpPlaintextColored}}}
\newcommand{\smUpCyphColored}{\mathord{\includegraphics[height=10ex]{pics/SmallUpCyphertextColored}}}


\newcommand{\smUpKeyOverlay}{\mathord{\includegraphics[height=10ex]{pics/SmallUpKeyOverlay}}}
\newcommand{\smUpMesOverlay}{\mathord{\includegraphics[height=10ex]{pics/SmallUpPlaintextOverlay}}}
\newcommand{\smUpCyphOverlay}{\mathord{\includegraphics[height=10ex]{pics/SmallUpCyphertextOverlay}}}


\newcommand{\smUpXorCyphColored}{\mathord{\includegraphics[height=10ex]{pics/SmallUpXorCyphertextColored}}}
\newcommand{\smUpXorMesColored}{\mathord{\includegraphics[height=10ex]{pics/SmallUpXorPlaintextColored}}}
\newcommand{\smUpXorCyphBlack}{\mathord{\includegraphics[height=10ex]{pics/SmallUpXorCyphertextBlack}}}
\newcommand{\smUpXorMesBlack}{\mathord{\includegraphics[height=10ex]{pics/SmallUpXorPlaintextBlack}}}



\newcommand{\exampleCyphertext}{\mathord{\includegraphics[height=15ex]{pics/exampleCyphertext}}}
\newcommand{\exampleCyphertextUp}{\mathord{\includegraphics[height=15ex]{pics/exampleCyphertextUp}}}
\newcommand{\exampleKey}{\mathord{\includegraphics[height=15ex]{pics/exampleKey}}}
\newcommand{\exampleKeyUp}{\mathord{\includegraphics[height=15ex]{pics/exampleKeyUp}}}
\newcommand{\examplePlaintext}{\mathord{\includegraphics[height=15ex]{pics/examplePlaintext}}}
\newcommand{\examplePlaintextUp}{\mathord{\includegraphics[height=15ex]{pics/examplePlaintextUp}}}
\newcommand{\exampleXorCyphertext}{\mathord{\includegraphics[height=15ex]{pics/exampleXorCyphertext}}}
\newcommand{\exampleXorPlaintext}{\mathord{\includegraphics[height=15ex]{pics/exampleXorPlaintext}}}




\newcommand{\exampleBadCyphertext}{\mathord{\includegraphics[height=15ex]{pics/exampleBadCyphertext}}}
\newcommand{\exampleBadCyphertextUp}{\mathord{\includegraphics[height=15ex]{pics/exampleBadCyphertextUp}}}
\newcommand{\exampleBadKey}{\mathord{\includegraphics[height=15ex]{pics/exampleBadKey}}}
\newcommand{\exampleBadKeyUp}{\mathord{\includegraphics[height=15ex]{pics/exampleBadKeyUp}}}
\newcommand{\exampleBadXorCyphertext}{\mathord{\includegraphics[height=15ex]{pics/exampleBadXorPlaintext}}}


\newcommand{\mesCircle}{\mathord{\includegraphics[height=15ex]{pics/circle}}}
\newcommand{\mesCross}{\mathord{\includegraphics[height=15ex]{pics/cross}}}
\newcommand{\mesSquare}{\mathord{\includegraphics[height=15ex]{pics/square}}}
\newcommand{\mesTriangle}{\mathord{\includegraphics[height=15ex]{pics/triangle}}}
\newcommand{\mesX}{\mathord{\includegraphics[height=15ex]{pics/X}}}


\title[Visual Encryption]{Visual Encryption}
\author{Giannis Moutsinas}
% \institute{University of Warwick}
\date{}

\begin{document}

\begin{frame}
  \titlepage
  \vfill

{\tiny
\doclicenseThis
}
\end{frame}

% Uncomment these lines for an automatically generated outline.
%\begin{frame}{Outline}
%  \tableofcontents
%\end{frame}

\begin{frame}{XOR operation}

The exclusive or (XOR) operation:
\begin{align*}
0+0=0\\
1+0=1\\
0+1=1\\
1+1=0
\end{align*}

\end{frame}

\begin{frame}{XOR encryption}

We can encrypt the message $11001100110011001100$ using the word 
$01110100100101011001$ as key by performing XOR bitwise.
\begin{equation*}
\begin{array}{rl}
  & \coloredOne \,\coloredOne\, \coloredZero \, \coloredZero \,\coloredOne\,\coloredOne\, \coloredZero \, \coloredZero \,\coloredOne\,\coloredOne\, \coloredZero \, \coloredZero \,\coloredOne\,\coloredOne\, \coloredZero \, \coloredZero \,\coloredOne\,\coloredOne\, \coloredZero \, \coloredZero \\
+ & \coloredZero \,\coloredOne\,\coloredOne\,\coloredOne\, \coloredZero \,\coloredOne\, \coloredZero \, \coloredZero \,\coloredOne\, \coloredZero \, \coloredZero \,\coloredOne\, \coloredZero \,\coloredOne\, \coloredZero \,\coloredOne\,\coloredOne\, \coloredZero \, \coloredZero \,\coloredOne\\ \hline
= &  \coloredOne \, \coloredZero \,\coloredOne\,\coloredOne\,\coloredOne\, \coloredZero \,\coloredOne\, \coloredZero \, \coloredZero \,\coloredOne\, \coloredZero \,\coloredOne\,\coloredOne\, \coloredZero \, \coloredZero \,\coloredOne\, \coloredZero \,\coloredOne\, \coloredZero \,\coloredOne
\end{array}
\end{equation*}

\pause
We can decrypt the message by performing bitwise XOR again.
\begin{equation*}
\begin{array}{rl}
  & \coloredOne\,\coloredZero\,\coloredOne\,\coloredOne\,\coloredOne\,\coloredZero\,\coloredOne\,\coloredZero\,\coloredZero\,\coloredOne\,\coloredZero\,\coloredOne\,\coloredOne\,\coloredZero\,\coloredZero\,\coloredOne\,\coloredZero\,\coloredOne\,\coloredZero\,\coloredOne \\
+ & \coloredZero \,\coloredOne\,\coloredOne\,\coloredOne\,\coloredZero\,\coloredOne\,\coloredZero\,\coloredZero\,\coloredOne\,\coloredZero\,\coloredZero\,\coloredOne\,\coloredZero\,\coloredOne\,\coloredZero\,\coloredOne\,\coloredOne\,\coloredZero\,\coloredZero\,\coloredOne\\ \hline
= & \coloredOne\,\coloredOne\,\coloredZero\,\coloredZero\,\coloredOne\,\coloredOne\,\coloredZero\,\coloredZero\,\coloredOne\,\coloredOne\,\coloredZero\,\coloredZero\,\coloredOne\,\coloredOne\,\coloredZero\,\coloredZero\,\coloredOne\,\coloredOne\,\coloredZero\,\coloredZero
\end{array}
\end{equation*}

\end{frame}

\begin{frame}{[title placeholder]}

Our goal is to perform XOR visually using transparencies.
We can use a black pixel ($\pxBlack$) to represent $1$ and a white pixel ($\pxWhite$) to represent $0$. Then the XOR operation with pixels becomes:
\begin{align*}
 \pxWhite+\pxWhite=\pxWhite \\
 \pxWhite+\pxBlack=\pxBlack \\
 \pxBlack+\pxWhite=\pxBlack \\
 \pxBlack+\pxBlack=\pxWhite
\end{align*}
The first $3$ hold true when we use transparencies, however the last one does not.

\end{frame}


\begin{frame}{[title placeholder]}

In order to do that we split each pixel into two. First we transform the pixels according to the following rule:
\begin{equation*}
\pxWhite \mapsto \RR \quad\text{ and }\quad \pxBlack \mapsto \LL.
\end{equation*}

\pause
Then we overlay the transparencies:
\begin{align*}
\LL + \LL = \LL \quad & \quad \RR + \RR = \RR \\
\LL + \RR = \pxBlack \quad & \quad \RR + \LL = \pxBlack
\end{align*}

\pause
Finally we can get the correct result if we transform the pixels using the following rule:
\begin{equation*}
\LL \mapsto \pxWhite \quad\text{ and }\quad \RR \mapsto \pxWhite.
\end{equation*}

\end{frame}





\begin{frame}{[title placeholder]}

The whole operation:
\begin{equation*}
\begin{array}{ccccccccccc}
&&& \pxWhite \mapsto \RR &&&&&& \RR \mapsto \pxWhite & \\
&&& \pxBlack \mapsto \LL &&&&&& \LL \mapsto \pxWhite & \\ \hline
\vspace{-0.5em}&&&&&&&&&& \\
\pxWhite & + & \pxWhite & \xRightarrow{\hspace*{3em}} & \RR & + & \RR &  \Longrightarrow & \RR & \xRightarrow{\hspace*{3em}} & \pxWhite \\
\pxWhite & + & \pxBlack & \xRightarrow{\hspace*{3em}} & \RR & + & \LL &  \Longrightarrow & \BB & \xRightarrow{\hspace*{3em}} & \pxBlack \\
\pxBlack & + & \pxWhite & \xRightarrow{\hspace*{3em}} & \LL & + & \RR &  \Longrightarrow & \BB & \xRightarrow{\hspace*{3em}} & \pxBlack \\
\pxBlack & + & \pxBlack & \xRightarrow{\hspace*{3em}} & \LL & + & \LL &  \Longrightarrow & \RR & \xRightarrow{\hspace*{3em}} & \pxWhite
\end{array}
\end{equation*}
\end{frame}



\begin{frame}{Example}

Let's do this for pictures with $4\times4$ pixels. We choose our message to be:
$$ \smMesBlack $$
and our key to be:
$$ \smKeyBlack $$



\end{frame}



\begin{frame}{Example: Encryption}

We change the message and the key according to the rule
\begin{equation*}
\pxWhite \mapsto \RR \quad\text{ and }\quad \pxBlack \mapsto \LL.
\end{equation*}
\begin{equation*}
\begin{array}{ccccc}
\smKeyBlack & \raisebox{4ex}{$\Rightarrow$} & \smUpKeyOverlay & \raisebox{4ex}{$\Rightarrow$} & \smUpKeyBlack \vspace{0.5em}\\
\smMesBlack & \raisebox{4ex}{$\Rightarrow$} & \smUpMesOverlay & \raisebox{4ex}{$\Rightarrow$} & \smUpMesBlack \vspace{0.5em} \\
% \smCyphBlack & \raisebox{4ex}{$\Rightarrow$} & \smUpCyphOverlay & \raisebox{4ex}{$\Rightarrow$} & \smUpCyphBlack
\end{array}
\end{equation*}
\end{frame}



\begin{frame}{Example: Encryption}

\begin{equation*}
\begin{array}{ccccc}
\smMesBlack & \raisebox{4ex}{+} & \smKeyBlack & \raisebox{4ex}{=} & \smCyphBlack \vspace{0.5em}\\
\Downarrow &  & \Downarrow& & \Uparrow \vspace{0.5em}\\
\smUpMesBlack & \raisebox{4ex}{+} & \smUpKeyBlack & \raisebox{4ex}{=} & \smUpXorMesBlack \vspace{0.5em} \\
\Downarrow &  & \Downarrow& & \Uparrow \vspace{0.5em}\\
\smUpMesColored & \raisebox{4ex}{+} & \smUpKeyColored & \raisebox{4ex}{=} & \smUpXorMesColored
\end{array}
\end{equation*}

\end{frame}

\begin{frame}{Example: Decryption}

We change the cyphertext and the key accoding to our rule
\begin{equation*}
\pxWhite \mapsto \RR \quad\text{ and }\quad \pxBlack \mapsto \LL.
\end{equation*}
\begin{equation*}
\begin{array}{ccccc}
\smCyphBlack & \raisebox{4ex}{$\Rightarrow$} & \smUpCyphOverlay & \raisebox{4ex}{$\Rightarrow$} & \smUpCyphBlack\\
\smMesBlack & \raisebox{4ex}{$\Rightarrow$} & \smUpMesOverlay & \raisebox{4ex}{$\Rightarrow$} & \smUpMesBlack \vspace{0.5em}
\end{array}
\end{equation*}
\end{frame}


\begin{frame}{Example: Decryption}

\begin{equation*}
\begin{array}{ccccc}
\smCyphBlack  &  & \smKeyBlack & & \smMesBlack \vspace{0.5em}\\
\Downarrow &  & \Downarrow& & \Uparrow \vspace{0.5em}\\
\smUpCyphBlack & \raisebox{4ex}{+} & \smUpKeyBlack & \raisebox{4ex}{$\Rightarrow$} & \smUpXorCyphBlack \vspace{0.5em} \\
\Downarrow &  & \Downarrow& & \Uparrow \vspace{0.5em}\\
\smUpCyphColored & \raisebox{4ex}{+} & \smUpKeyColored & \raisebox{4ex}{$\Rightarrow$} & \smUpXorCyphColored
\end{array}
\end{equation*}


\end{frame}



\begin{frame}{Big example: Encryption}

\begin{equation*}
\begin{array}{ccccc}
\examplePlaintext &  & \exampleKey & & \exampleCyphertext \vspace{0.5em}\\
\Downarrow &  & \Downarrow& & \Uparrow \vspace{0.5em}\\
\examplePlaintextUp & \raisebox{7ex}{+} & \exampleKeyUp & \raisebox{7ex}{$\Rightarrow$} & \exampleXorPlaintext
\end{array}
\end{equation*}

\end{frame}



\begin{frame}{Big example: Decryption}

\begin{equation*}
\begin{array}{ccccc}
\exampleCyphertext & & \exampleKey & & \examplePlaintext \vspace{0.5em}\\
\Downarrow &  & \Downarrow& & \Uparrow \vspace{0.5em}\\
\exampleCyphertextUp & \raisebox{7ex}{+} & \exampleKeyUp & \raisebox{7ex}{$\Rightarrow$} & \exampleXorCyphertext
\end{array}
\end{equation*}

\end{frame}



\begin{frame}{Dig example: Bad key}

\begin{equation*}
\begin{array}{ccccc}
\examplePlaintext &  & \exampleBadKey & & \exampleBadCyphertext \vspace{0.5em}\\
\Downarrow &  & \Downarrow& & \Uparrow \vspace{0.5em}\\
\examplePlaintextUp & \raisebox{7ex}{+} & \exampleBadKeyUp & \raisebox{7ex}{$\Rightarrow$} & \exampleBadXorCyphertext
\end{array}
\end{equation*}

\end{frame}


\begin{frame}{Possible messages}

\begin{equation*}
\begin{array}{ccccc}
\mesCircle &  & \mesCross & & \mesSquare
\end{array}
\end{equation*}

\begin{equation*}
\begin{array}{ccccc}
\mesTriangle &  & \mesX
\end{array}
\end{equation*}

\end{frame}


\end{document}
